\documentclass[12pt]{article}
\usepackage[margin=1in]{geometry}
\usepackage{graphicx}
\usepackage{hyperref}
\usepackage{enumitem}
\usepackage{booktabs}
\usepackage{longtable}
\usepackage{titlesec}
\usepackage{fancyhdr}
\usepackage{datetime}
\usepackage{url}
\usepackage{multicol}
\usepackage{setspace}
\usepackage{caption}
\usepackage{float}

\hypersetup{
	colorlinks=true,
	linkcolor=blue,
	urlcolor=blue,
	pdftitle={LAL202 International Trade Assignment},
	pdfauthor={Paritosh Lahre},
	pdfsubject={International Trade},
	pdfkeywords={India, Trade, WTO, FTAs, Exports, Imports, Trade War}
}

\pagestyle{fancy}
\fancyhf{}
\lhead{LAL202 International Trade}
\rhead{Paritosh Lahre (ID: 12341550)}
\cfoot{\thepage}

\titleformat{\section}
{\normalfont\Large\bfseries}{\thesection}{1em}{}

\titleformat{\subsection}
{\normalfont\large\bfseries}{\thesubsection}{1em}{}

\begin{document}
	
	\begin{center}
		\LARGE\textbf{Assignment: LAL202 International Trade} \\
		\vspace{0.2cm}
		\large\textbf{Name:} Paritosh Lahre \\
		\textbf{Student ID:} 12341550 \\
	\end{center}
	
	\vspace{1cm}
	
	\section{Functions of the World Trade Organization (WTO)}
	
	The World Trade Organization (WTO) serves as the principal international body governing global trade. Its primary functions include:
	
	\begin{enumerate}[label=\arabic*.]
		\item \textbf{Administering WTO Trade Agreements:} Ensures the implementation and monitoring of agreed-upon trade rules among member countries.
		\item \textbf{Forum for Trade Negotiations:} Provides a platform for member nations to negotiate trade agreements and resolve trade-related issues.
		\item \textbf{Handling Trade Disputes:} Offers a structured process for resolving trade disputes between countries.
		\item \textbf{Monitoring National Trade Policies:} Conducts regular reviews of individual countries' trade policies to ensure transparency and adherence to WTO rules.
		\item \textbf{Technical Assistance and Training:} Provides support to developing countries to help them build their trade capacity.
		\item \textbf{Cooperation with Other International Organizations:} Collaborates with institutions like the IMF and World Bank to ensure coherence in global economic policy-making.
		\item \textbf{Ensuring Non-Discrimination:} Upholds the principles of Most-Favored-Nation (MFN) and National Treatment to promote fair competition.
		\item \textbf{Promoting Economic Stability:} Aims to reduce trade barriers and foster economic growth and stability worldwide.
	\end{enumerate}
	
	\section{Eight Important Free Trade Agreements (FTAs) of India}
	
	India has entered into several FTAs to enhance its trade relations. Below are eight significant agreements:
	
	\begin{longtable}{@{}p{4cm}p{4cm}p{6cm}@{}}
		\toprule
		\textbf{Agreement} & \textbf{Members/Countries} & \textbf{Purpose} \\
		\midrule
		\endhead
		\textbf{India–ASEAN FTA} & India and 10 ASEAN countries & To promote trade and economic integration between India and Southeast Asian nations. \\
		\textbf{India–Japan CEPA} & India and Japan & To eliminate tariffs and promote investment and services trade. \\
		\textbf{India–South Korea CEPA} & India and South Korea & To enhance bilateral trade and investment flows. \\
		\textbf{India–Sri Lanka FTA} & India and Sri Lanka & To promote free trade in goods and services. \\
		\textbf{India–Bhutan Trade Agreement} & India and Bhutan & To facilitate duty-free trade and economic cooperation. \\
		\textbf{India–Nepal Trade Treaty} & India and Nepal & To promote trade and economic integration. \\
		\textbf{India–UAE CEPA} & India and United Arab Emirates & To boost trade in goods, services, and investments. \\
		\textbf{India–Australia ECTA} & India and Australia & To enhance bilateral trade and economic cooperation. \\
		\bottomrule
	\end{longtable}
	
	\section{India's Major Export and Import Partners}
	
	\subsection{Top 5 Export Partners (2023)}
	
	\begin{enumerate}[label=\arabic*.]
		\item \textbf{United States} - \$80.23 billion
		\item \textbf{United Arab Emirates} - \$31.32 billion
		\item \textbf{Netherlands} - \$18.50 billion
		\item \textbf{China} - \$15.08 billion
		\item \textbf{Bangladesh} - \$13.83 billion
	\end{enumerate}
	
	\subsection{Top 5 Import Partners (2023)}
	
	\begin{enumerate}[label=\arabic*.]
		\item \textbf{China} - \$102.25 billion
		\item \textbf{United Arab Emirates} - \$53.85 billion
		\item \textbf{United States} - \$51.77 billion
		\item \textbf{Saudi Arabia} - \$46.18 billion
		\item \textbf{Russia} - \$40.62 billion
	\end{enumerate}
	
	\section{Current Developments in the U.S.–China Trade War (As of April 2025)}
	
	The U.S.–China trade tensions have seen several developments:
	
	\begin{itemize}
		\item \textbf{China's Tariff Adjustments:} China has exempted certain U.S. semiconductors from its 125\% retaliatory tariffs and is considering further exemptions on medical equipment and industrial chemicals.
		\item \textbf{China's Economic Measures:} President Xi Jinping announced plans to boost domestic demand and support enterprises to counteract the trade war's impact.
		\item \textbf{U.S. Policy Uncertainty:} President Trump's inconsistent statements on tariffs have created global economic uncertainty.
		\item \textbf{Market Reactions:} These developments have led to cautious optimism in Asian markets, with some recovery in stock indices and currencies.
	\end{itemize}
	
\end{document}
